% dvips -t letter hw_ch2.dvi -o hw_ch2.ps ; ps2pdf hw_ch2.ps
\documentclass[11pt,titlepage,fleqn]{article}

\usepackage{amsmath}
\usepackage{amssymb}
\usepackage{latexsym}
\usepackage[round]{natbib}
%\usepackage{epsfig}
\usepackage{graphicx}
\usepackage{bm}

\usepackage{url}
\usepackage{color}

%--------------------------------------------------------------
%       SPACING COMMANDS (Latex Companion, p. 52)
%--------------------------------------------------------------

\usepackage{setspace}    % double-space or single-space
\usepackage{xspace}

\renewcommand{\baselinestretch}{1.2}

\textwidth 460pt
\textheight 690pt
\oddsidemargin 0pt
\evensidemargin 0pt

% see Latex Companion, p. 85
\voffset     -50pt
\topmargin     0pt
\headsep      20pt
\headheight   15pt
\headheight    0pt
\footskip     30pt
\hoffset       0pt

\include{carlcommands}

\graphicspath{
  {./figures/}
}

\newcommand{\repodir}{{\tt inverse}}

\newcommand{\howmuchtime}{Approximately how much time {\em outside of class and lab time} did you spend on this problem set? Feel free to suggest improvements here.}

% provide space for students to write their solutions
\newcommand{\vertgap}{\vspace{1cm}}

\newcommand{\Ucolor}{\textcolor{red}{\bU}}
\newcommand{\Vcolor}{\textcolor{blue}{\bV}}

\newcommand{\Gcolor}{\textcolor{red}{\bU}\bS\textcolor{blue}{\bV^T}}
\newcommand{\Gpcolor}{\textcolor{red}{\bU_p}\,\bS_p\textcolor{blue}{\bV_p^T}}
\newcommand{\Gdcolor}{\textcolor{blue}{\bV_p}\,\bS_p^{-1}\textcolor{red}{\bU_p^T}}
\newcommand{\GcolorT}{\textcolor{blue}{\bV}\bS^T\textcolor{red}{\bU^T}}
\newcommand{\GpcolorT}{\textcolor{blue}{\bV_p}\,\bS_p\textcolor{red}{\bU_p^T}}
\newcommand{\GdcolorT}{\textcolor{red}{\bU_p}\,\bS_p^{-1}\textcolor{blue}{\bV_p^T}}

\newcommand{\blank}{xxxx}


\renewcommand{\baselinestretch}{1.1}

\newcommand{\tfile}{{\tt hw\_ch2\_ch2p1.ipynb}}
%\newcommand{\tfileA}{{\tt ex2p1_ex2p2.ipynb}}
%\newcommand{\tfileB}{{\tt ex3p1.ipynb}}

%--------------------------------------------------------------
\begin{document}
%-------------------------------------------------------------

\begin{spacing}{1.2}
\centering
{\large \bf Problem Set 6: Aster Ch. 2 [ch2]} \\
\cltag\ \\
Assigned: March 14, 2022 --- Due: March 23, 2022 \\
Last compiled: \today
\end{spacing}

%------------------------

\subsection*{Instructions}

\begin{itemize}
\item These problems are from \citet{Aster}, Chapters 2 and 3. Be sure you understand Examples 2-1, 2-2, and 3-1, featured in \verb+lab_ch2.pdf+.
\item Wherever it says ``calculate \ldots'' or ``compute \ldots,'' please list the numerical output.
\item List units for all answers.
\end{itemize}

%\pagebreak
\subsection*{Problem 1 (3.0). Aster 2-1 modified.}

A seismic profiling experiment is performed where the first arrival times of seismic energy from a mid-crustal refractor are observed at distances (in kilometers) of $\bx = (x_1, \ldots, x_m)$ from the source, and are found to be (in seconds after the source origin time) $\bt = (t_1, \ldots, t_m)$. The vectors $\bx$ and $\bt$ are listed in the template program \tfile.

A two-layer, flat Earth structure gives the mathematical model
%
\begin{equation}
t_i = t_0 + s_2x_i,
\label{twolayer}
\end{equation}
%
where the intercept time $t_0$ depends on the thickness and slowness of the upper layer, and $s_2$ is the slowness of the lower layer. The estimated noise in the first arrival time measurements is believed to be independent and normally distributed with the expected value 0 and standard deviation $\sigma = 0.1$~s. (Note: Up until problem~1-\ref{monte}, you are not going to generate any errors. The errors are unknown and ``buried'' within the arrival time measurements that are provided in this problem.)
%
\begin{enumerate}
\item (0.0) Write \refEq{twolayer} as a forward model $\bd = \bG\bem$ in matrix-vector schematic form.

\item (0.1) Write the weighted least squares solution (eq. 2.17) in terms of $\covd$ instead of $\bW$.

\item (0.7) 
\begin{enumerate}
\item (0.3) Calculate the weighted least squares solution for the model parameters $t_0$ and~$s_2$.

Note: It is probably easiest to use the transformed variables $\bd_w$ and $\bG_w$.
\item (0.4) Plot the data, the data uncertainties, and the predictions for the fitted model as a dashed line.
\end{enumerate}


\item (0.2) Calculate the posterior model parameter covariance matrix $\cpost$ and the corresponding correlation matrix $\rhopost$. List the entries with at least 3 significant figures.

How will the correlations be manifested in the general appearance of the error ellipsoid in $(t_0,s_2)$ space?

\pagebreak
\item (0.3) Calculate conservative 95\% confidence intervals for $t_0$ and $s_2$.
%
\begin{enumerate}
\item (0.1) What value of $\Delta$ should you use?

Hint: Use \verb+chi2inv()+ to get $\Delta^2$; use $n$ degrees of freedom, as in Example 2.2 and described on p.~34--35. (In the the other parts of Problem 1, you should use $m-n$ for the number of degrees of freedom, when needed.)

\item (0.1) List the confidence intervals for $t_0$ and $s_2$.
\item (0.1) What is the corresponding interval for $v_2 = 1/s_2$?
\end{enumerate}

\item (0.3) Calculate and plot the 95\% confidence error ellipsoid in the $(t_0, s_2)$ plane.

Hint: Use \verb+plot_ellipse()+, which has inputs $\Delta^2$, $\bC$, and $\bem$.

Note: To check your result, you can display the 2000 estimated models from the following problem.

\item (0.2) Compute $\chi^2_{\rm obs}$ and the $p$-value for the solution.

\item (1.0) Evaluate the value of $\chi^2_{\rm obs}$ for 2000 Monte Carlo simulations using the data prediction from your model perturbed by noise that is consistent with the data assumptions. Compare a histogram of these $\chi^2_{\rm obs}$ values with the theoretical $\chi^2$ distribution for the correct number of degrees of freedom.
%You may find the library function \verb+chi2pdf.m+ to be useful here.
You may find the \verb+chi_2.pdf()+ to be useful here.

\label{monte}

\item (0.2) Are your $p$-value and $\chi^2_{\rm obs}$ for the solution consistent with the theoretical modeling and the data set? If not, explain what might be wrong.

\end{enumerate}

%--------------------------------------------------------------

%\pagebreak
\subsection*{Problem 2 (2.0). Aster 2-2.}

Suppose that the data errors have an multivariate normal distribution with expected value $\bzero$ and a covariance matrix $\covd$. (We are not making any assumptions about $\covd$ being diagonal.) It can be shown that the likelihood function is then
%
\begin{equation}
L(\bem) = k \exp \left[ -f(\bem) \right]
\end{equation}
%
where
%
\begin{equation}
f(\bem) = (\bG\bem - \bd)^T \covdi (\bG\bem - \bd).
\label{fm}
\end{equation}
%
\begin{enumerate}
\item (0.2) Show that the maximum likelihood estimate can be obtained by solving the minimization problem,
%
\begin{equation}
\min \left[ (\bG\bem - \bd)^T \covdi (\bG\bem - \bd) \right]
\label{min}
\end{equation}

%-------

\item (1.2) Show that \refEq{min} will lead to the system of equations
%
\begin{equation}
\bG^T \covdi \bG \bem = \bG^T \covdi \bd
\end{equation}
%
Your derivation should not involve any more variables than those in \refEq{fm}.

Hint: Review \verb+notes_taylor.pdf+.

%-------

\item (0.3) \ptag\ Show that \refEq{min} is equivalent to the linear least squares problem
%
\begin{equation}
\min\left\|  \covd^{-1/2}\bG\bem - \covd^{-1/2}\bd  \right\|_2^2
\end{equation}
%
where $\covd^{-1/2}$ is the matrix square root, \ie $\covdi = \covd^{-1/2} \covd^{-1/2}$.

%-------

\item (0.3) \ptag\ The Cholesky factorization of $\covdi$ can be used instead of the matrix square root. Show that \refEq{min} is equivalent to the linear least squares problem
%
\begin{equation}
\min\left\|  \bR\bG\bem - \bR\bd  \right\|_2^2
\end{equation}
%
where $\bR$ is the Cholesky factor: $\covdi = \bR^T\bR$.

\end{enumerate}

%--------------------------------------------------------------

%\pagebreak
\subsection*{Problem 3 (1.5). Aster 2-5 modified.}

In this problem you will use linear regression to fit a polynomial of the form\footnote{The $\nparm = 20$ coefficients are the unknown model parameters.}
%
\begin{equation}
y_i = a_0 + a_1x_i + a_2x_i^2 + \cdots + a_{19}x_i^{19}
\label{yi}
\end{equation}
%
to the $\ndata=21$ noise-free data points
%
\begin{equation*}
(x_i,y_i) = (-1,-1), (-0.9,-0.9), (-0.8,-0.8), \ldots, (0.9,0.9), (1,1)
\end{equation*}
%
\begin{itemize}
\item Do not use any pseudo-inverse code functions (like \verb+pinv+) in this problem; these will come in Chapter~3.

\item Do not use $\covd$ in this problem.
\end{itemize}

%---------------

\begin{enumerate}
\item (0.3) Based on \refEq{yi}, write out the least squares problem $\bG\bem = \dvec$ in schematic matrix-vector notation (list variables, not numbers).

%--------

\item (0.5) 

\begin{enumerate}
\item (0.1) Write the generic expression for the solution $\bem_{\rm lsq}$ to the least squares problem.
\item (0.2) Use the QR factorization of $\bG$ to write the expression for the solution $\bem_{\rm qr}$ for \makebox{$\bG\bem = \dvec$}. Hint: \citet[][Appendix A]{Aster}.

\item (0.2) Use the SVD factorization of $\bG$ to write the expression for the solution $\bem_{\rm svd}$ for \makebox{$\bG\bem = \dvec$}. Hint: your expressions should have the $p$ subscripts.
\end{enumerate}

\item (0.7)

\begin{enumerate}
\item (0.1) What are the numerical entries of true model $\bem_{\rm true}$?

\item (0.2) Plot the true model and your least-squares estimated model, $\bem_{\rm lsq}$. The $x$-axis can be the index $i = 0,1,\ldots,19$ of $a_i$.

\item (0.2) Plot the discrete data points $(x_i,y_i)$. Superimpose the predictions from your least-squares model as a continuous function of $x$.

%\item (0.2) Calculate $\bem_{\rm lsq}$, $\bem_{\rm qr}$, and $\bem_{\rm svd}$, and list the output for convenient comparison (like \verb+[mtrue mlsq mqr msvd]+).

\item (0.2) Discuss the performance of the LSQ, QR, and SVD approaches to solving this problem.

\end{enumerate}

%\item (0.3) \ptag\ To demonstrate the instability of the solution, compute the least-squares solution \verb+mlsqmod = inv(G'*G)*G'*y+. (Note \footnote{This equation is based on a non-square $\bG$, but of course our $\bG$ is square. Do not motify our $\bG$.})
%
%\begin{enumerate}
%\item How does this model and its predictions compare with the previous results?
%\item Explain why $\bem_{\rm lsqmod}$ differs from $\bem_{\rm true}$.
%\item What do you expect will happen to your model vectors if you add noise to the data, \ie the $y$ values?
%\end{enumerate}

\end{enumerate}

%--------------------------------------------------------------

\subsection*{Problem 4 (1.5). Aster 3-1.}

The pseudoinverse of a matrix $\bG$ was originally defined by Moore and Penrose as the unique matrix $\bG^{\dagger}$ with the properties
%
\begin{enumerate}
\item (0.4) $\bG\bG^{\dagger}\bG = \bG$
\item (0.4) $\bG^{\dagger}\bG\bG^{\dagger} = \bG^{\dagger}$
\item (0.4) $\left( \bG\bG^{\dagger} \right)^T = \bG\bG^{\dagger}$
\item (0.3) $\left( \bG^{\dagger}\bG \right)^T = \bG^{\dagger}\bG$
\end{enumerate}
%
Show that $\bG^{\dagger} = \bV_p \bS_p^{-1} \bU_p^T$ satisfies these four properties.

%--------------------------------------------------------------

\subsection*{Problem 5 (2.0). Aster 3-2.}

Example~3.1 in \citet{Aster}---which we explored in \verb+lab_ch2.pdf+---presented a resolution test for a ``spike model'' in a tomography problem.
Another resolution test commonly performed in tomography studies is a {\bf checkerboard test}, which consists of using a test model composed of alternating positive and negative perturbations. Perform a checkerboard test on the tomography problem in Example~3.1 using the test model
%
\begin{equation*}
\bem_{\rm checker} = \left[ \begin{array}{r}
     -1  \\  1 \\ -1 \\  1  \\ -1 \\  1 \\ -1  \\  1 \\ -1
\end{array} \right]
\end{equation*}
%
\refFig{fig:index} shows how the model parameters are indexed. In physical space, the checkerboard model is represented by the $3 \times 3$ matrix
%
\begin{equation*}
\left[ \begin{array}{rrr}
     -1  &  1 & -1   \\
      1  & -1 &  1   \\
     -1  &  1 & -1   \\
\end{array} \right]
\end{equation*}
%
See \refApp{sec:pert}.

%----------------------

\begin{enumerate}

\item (0.0) Recall from lab:
%
\begin{enumerate}
\item What is the forward model in this problem?
\item Examine \refFig{fig:index}. Write the system of equations for $t_1, \ldots, t_8$.
\item Convince yourself that the $\bG$ provided for this problem is correct.
\end{enumerate}

\item (0.2) What does the row-reduced form of $\bG$ (\verb+rref(G)+) for this problem tell you about the rank of $\bG$ and the type of solution we might expect?

\item (1.2) 
\begin{enumerate}
\item (0.4) List the values for the recovered model $\bem_{\dagger} = \bR_{\rm M}\bem_{\rm checker}$. Plot the vector.
\item (0.2) List the values for the difference $\bem_{\dagger}-\bem_{\rm checker}$. \\
What parameters are perfectly recovered? Plot the vector.
\item (0.2) What is the formula for the covariance matrix $\cpost$ associated with $\bem_{\dagger}$? \\ Assume that $\covd = \sigma^2\bI$.
\item (0.2) Compute the covariance matrix $\cpost$ using $\sigma = 1$~s. \\
List the standard deviations associated with $\cpost$.
\item (0.2) \ptag\ Compute and plot the correlation matrix $\rhopost$. List the correlation coefficient between the 2nd and 4th model parameters.
\end{enumerate}

\begin{itemize}
\item Do not use the \verb+pinv+ or \verb+inv+ commands.
\item All model vectors should be plotted as $3 \times 3$ matrices.
\item Use \verb+caxis(p*[-1 1])+ (where \verb+p+ is a sensible choice), so that the zero values are easy to identify.
\end{itemize}

\item (0.4) Interpret the pattern of differences $\bem_{\dagger}-\bem_{\rm checker}$ in the context of the geometry of the experiment (\refFig{fig:index}).

\item (0.2) If any block values are recovered exactly, does this imply perfect resolution for these model parameters?
\end{enumerate}

%------------------------

%\pagebreak
\subsection*{Problem} \howmuchtime\

\pagebreak

%-------------------------------------------------------------
\nocite{Aster}
\bibliography{carl_abbrev,carl_main,carl_source,carl_him,carl_alaska}
%-------------------------------------------------------------

\appendix

\section{Aside: model parameters as perturbations}
\label{sec:pert}

The tomography problem was introduced in Example~1.12 with the unknown entries of the model vector being slowness values $s_k = 1/v_k$, where $v_k > 0$ is the velocity of the medium. In \citet{Aster} Example 3.1, there are negative entries in $\bem$, which is not permitted for slowness. The entries of $\bem$ should be thought of as {\em perturbations} in slowness from some uniform value, $s_0$. Then the entries in the data vector should be thought of as {\em perturbations} in travel time from the reference travel time, $\bd_0 = \bG\bem_0$, where $\bem_0$ has $s_0$ in each entry. We can rewrite our problem as
%
\begin{eqnarray}
\bG\bem &=& \bd
\\
\bG\bem_0 &=& \bd_0
\\
\bG(\bem - \bem_0) &=& \bd - \bd_0
\\
\bG\bDelta\bem &=& \bDelta\bd
\end{eqnarray}
%
In this case, there is nothing wrong with $\bDelta\bem < 0$ or $\bDelta\bd < 0$. We should think of the checkerboard as $\bDelta\bem$ and the entries of the data vector as differential times.

%---------------------------------

%\clearpage\pagebreak

\begin{figure}[h]
\centering
\begin{tabular}{lc}
{\bf(a)} & \includegraphics[width=8cm]{hw_ch3p2_index.eps} \\
{\bf(b)} & \includegraphics[width=9cm]{aster_tomo_rays.eps} \\
\end{tabular}
\caption[]
{{
Setup for the tomography problem.
(a) Indexing of the $\nparm=9$ model parameters.
(Note that here the checkerboard pattern is only plotted to illuminate the boundaries of the cells.)
(b) Ray paths for the $\ndata=8$ measurements.
\label{fig:index}
}}
\end{figure} 

%-------------------------------------------------------------
\end{document}
%-------------------------------------------------------------
