% dvips -t letter lab_svdgeom.dvi -o lab_svdgeom.ps ; ps2pdf lab_svdgeom.ps
\documentclass[11pt,titlepage,fleqn]{article}

\usepackage{amsmath}
\usepackage{amssymb}
\usepackage{latexsym}
\usepackage[round]{natbib}
%\usepackage{epsfig}
\usepackage{graphicx}
\usepackage{bm}

\usepackage{url}
\usepackage{color}

%--------------------------------------------------------------
%       SPACING COMMANDS (Latex Companion, p. 52)
%--------------------------------------------------------------

\usepackage{setspace}    % double-space or single-space
\usepackage{xspace}

\renewcommand{\baselinestretch}{1.2}

\textwidth 460pt
\textheight 690pt
\oddsidemargin 0pt
\evensidemargin 0pt

% see Latex Companion, p. 85
\voffset     -50pt
\topmargin     0pt
\headsep      20pt
\headheight   15pt
\headheight    0pt
\footskip     30pt
\hoffset       0pt

\include{carlcommands}

\graphicspath{
  {./figures/}
}

\newcommand{\repodir}{{\tt inverse}}

\newcommand{\howmuchtime}{Approximately how much time {\em outside of class and lab time} did you spend on this problem set? Feel free to suggest improvements here.}

% provide space for students to write their solutions
\newcommand{\vertgap}{\vspace{1cm}}

\newcommand{\Ucolor}{\textcolor{red}{\bU}}
\newcommand{\Vcolor}{\textcolor{blue}{\bV}}

\newcommand{\Gcolor}{\textcolor{red}{\bU}\bS\textcolor{blue}{\bV^T}}
\newcommand{\Gpcolor}{\textcolor{red}{\bU_p}\,\bS_p\textcolor{blue}{\bV_p^T}}
\newcommand{\Gdcolor}{\textcolor{blue}{\bV_p}\,\bS_p^{-1}\textcolor{red}{\bU_p^T}}
\newcommand{\GcolorT}{\textcolor{blue}{\bV}\bS^T\textcolor{red}{\bU^T}}
\newcommand{\GpcolorT}{\textcolor{blue}{\bV_p}\,\bS_p\textcolor{red}{\bU_p^T}}
\newcommand{\GdcolorT}{\textcolor{red}{\bU_p}\,\bS_p^{-1}\textcolor{blue}{\bV_p^T}}

\newcommand{\blank}{xxxx}


\renewcommand{\baselinestretch}{1.1}

\newcommand{\tfile}{{\tt lab\_svdgeom.ipynb}}

% change the figures to ``Figure L3'', etc
\renewcommand{\thefigure}{L\arabic{figure}}
\renewcommand{\thetable}{L\arabic{table}}
\renewcommand{\theequation}{L\arabic{equation}}
\renewcommand{\thesection}{L\arabic{section}}

%--------------------------------------------------------------
\begin{document}
%-------------------------------------------------------------

\begin{spacing}{1.2}
\centering
{\large \bf Lab Exercise: Geometry of the singular value decomposition [svdgeom]} \\
\cltag\ \\
Last compiled: \today
\end{spacing}

%------------------------

\subsection*{Instructions}

%The singular value decomposition is featured in \citet{Aster}.
The objective of this lab is to reinforce the geometrical concept of SVD \citep{TrefethenBau} presented in \verb+notes_svd.pdf+. {\bf You will need a ruler.}

Following \verb+notes_svd.pdf+, \refFig{fig:2D} shows four examples of different ($2 \times 2$) $\bG$ representing the SVD.

\begin{enumerate}
\item Using \refFig{fig:ex1}, estimate $\Ucolor$ (\textcolor{red}{data space}), $\Vcolor$ (\textcolor{blue}{model space}), $\bS$, and then calculate $\bG$.
%(The prime notation is to distinguish your estimated matrices from the true matrices.)

If your entries of $\bG$ are not close to integer values, please try again.

Hint: Use a ruler, then make sure that your vectors are normalized. See Note \footnote{
Measuring the coordinates of two basis vectors will not guarantee that the two vectors are orthogonal. For this, you could use a rotation matrix
%
\begin{equation}
\bR_\theta = \left[
\begin{array}{rr}
\cos\theta & -\sin\theta \\ 
\sin\theta & \cos\theta
\end{array}
\right]
\end{equation}
%
This can be implemented as a function. Provide a $\theta$, then the columns of $\bR_\theta$ will be the rotated standard basis vectors.

To get $\Ucolor$ and $\Vcolor$, use $\bR_\theta$. You may also want to use
%
\begin{equation}
\bZ = \left[
\begin{array}{rr}
0 & 1 \\ 
1 & 0
\end{array}
\right],
\end{equation}
%
since $\bA\bZ$ will swap the two column vectors of $\bA$.
}

\item Make sure you have the true $\bG$ before proceeding.

\item In the {\bf top} of \refFig{fig:ex1}, sketch $\be_1$ and $\be_2$ in the left plot. \\
Sketch $\bG\be_1$ and $\bG\be_2$ in the right plot.

\item The {\bf bottom} of \refFig{fig:ex1} represents the mapping from $\bG^{\dagger}\Ucolor$. Sketch the following (note that you do not need $\bG^{\dagger}$ for the sketch):
%
\begin{itemize}
\item $\bu_1$ and $\bu_2$ in the left plot
\item appropriately scaled versions of $\bv_1$ and $\bv_2$ in the right plot
\item the ellipse representing $\bG^{\dagger}\Ucolor$ in the right plot
\end{itemize}

\item Open \tfile\ and execute the first cell. What does the function \verb+svdmat()+ do (see \verb+lib_geos.py+)?

\item Using \verb+svdmat+, calculate $\Ucolor$, $\Vcolor$, and $\bS$, and then calculate $\bG^{\dagger}$. \\
Sketch $\be_1$ and $\be_2$ in the lower left plot. \\
Sketch $\bG^{\dagger}\be_1$ and $\bG^{\dagger}\be_2$ in the lower right plot.

\item Does $\bG^{\dagger} = \bG^{-1}$? Why?

\item Compute the eigenbasis of $\bG$ and then sketch eigenvectors $\{\bh_1,\bh_2\}$ and $\{\bG\bh_1,\bG\bh_2\}$ in your top plots.

Note: You can't sketch the eigenvectors if they are complex.

\item Get some more practice with three more examples in \refFigab{fig:ex2}{fig:ex4}.

\item Run \tfile. By default, the code will generate a random $\bG$ with all have integer entries between $-2$ and 2.

Try a few different $\bG$ to see what happens. \\
You may want to try the four examples in \verb+notes_svd.pdf+. \\
Check that your $\bG$ will generate \refFig{fig:ex1}.

\item The top of \refFig{fig:ex1} represents $\Vcolor \rightarrow \bG \rightarrow \Ucolor$.

Write some code to generate a plot representing $\Ucolor \rightarrow \bG^{\dagger} \rightarrow \Vcolor$. Stick to the coloring convention for $\Ucolor$ and $\Vcolor$.

\end{enumerate}

%-------------------------------------------------------------
\bibliography{carl_abbrev,carl_main}
%-------------------------------------------------------------

\clearpage\pagebreak
\begin{figure}
\hspace{-1cm}
\includegraphics[width=18cm]{svd_2D_both_5}
\caption[]
{{
Example 1.
\label{fig:ex1}
}}
\end{figure}

\clearpage\pagebreak
\begin{figure}
\hspace{-1cm}
\includegraphics[width=18cm]{svd_2D_both_6}
\caption[]
{{
Example 2.
\label{fig:ex2}
}}
\end{figure}

\clearpage\pagebreak
\begin{figure}
\hspace{-1cm}
\includegraphics[width=18cm]{svd_2D_both_7}
\caption[]
{{
Example 3.
\label{fig:ex3}
}}
\end{figure}

\clearpage\pagebreak
\begin{figure}
\hspace{-1cm}
\includegraphics[width=18cm]{svd_2D_both_8}
\caption[]
{{
Example 4.
\label{fig:ex4}
}}
\end{figure}

\clearpage\pagebreak
\pagestyle{empty}
\begin{figure}
\centering
\begin{tabular}{c}
\includegraphics[width=13cm]{svd_2D_5} \\
\includegraphics[width=13cm]{svd_2D_6} \\
\includegraphics[width=13cm]{svd_2D_7} \\
\includegraphics[width=13cm]{svd_2D_8} 
\end{tabular}
\caption[]
{{
The matrix $\bG = \Ucolor\,\bS\,\Vcolor^T$ transforms the orthonormal basis vectors \textcolor{blue}{$\{\bv_i\}$} to the orthogonal vectors \textcolor{red}{$\{s_i\bu_i\}$}: \makebox{$\bG\Vcolor = \Ucolor\,\bS$}.
\label{fig:2D}
}}
\end{figure}

%-------------------------------------------------------------
\end{document}
%-------------------------------------------------------------
