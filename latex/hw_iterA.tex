% dvips -t letter hw_iter.dvi -o hw_iter.ps ; ps2pdf hw_iter.ps
\documentclass[11pt,titlepage,fleqn]{article}

\usepackage{amsmath}
\usepackage{amssymb}
\usepackage{latexsym}
\usepackage[round]{natbib}
%\usepackage{epsfig}
\usepackage{graphicx}
\usepackage{bm}

\usepackage{url}
\usepackage{color}

%--------------------------------------------------------------
%       SPACING COMMANDS (Latex Companion, p. 52)
%--------------------------------------------------------------

\usepackage{setspace}    % double-space or single-space
\usepackage{xspace}

\renewcommand{\baselinestretch}{1.2}

\textwidth 460pt
\textheight 690pt
\oddsidemargin 0pt
\evensidemargin 0pt

% see Latex Companion, p. 85
\voffset     -50pt
\topmargin     0pt
\headsep      20pt
\headheight   15pt
\headheight    0pt
\footskip     30pt
\hoffset       0pt

\include{carlcommands}

\graphicspath{
  {./figures/}
}

\newcommand{\repodir}{{\tt inverse}}

\newcommand{\howmuchtime}{Approximately how much time {\em outside of class and lab time} did you spend on this problem set? Feel free to suggest improvements here.}

% provide space for students to write their solutions
\newcommand{\vertgap}{\vspace{1cm}}

\newcommand{\Ucolor}{\textcolor{red}{\bU}}
\newcommand{\Vcolor}{\textcolor{blue}{\bV}}

\newcommand{\Gcolor}{\textcolor{red}{\bU}\bS\textcolor{blue}{\bV^T}}
\newcommand{\Gpcolor}{\textcolor{red}{\bU_p}\,\bS_p\textcolor{blue}{\bV_p^T}}
\newcommand{\Gdcolor}{\textcolor{blue}{\bV_p}\,\bS_p^{-1}\textcolor{red}{\bU_p^T}}
\newcommand{\GcolorT}{\textcolor{blue}{\bV}\bS^T\textcolor{red}{\bU^T}}
\newcommand{\GpcolorT}{\textcolor{blue}{\bV_p}\,\bS_p\textcolor{red}{\bU_p^T}}
\newcommand{\GdcolorT}{\textcolor{red}{\bU_p}\,\bS_p^{-1}\textcolor{blue}{\bV_p^T}}

\newcommand{\blank}{xxxx}


\renewcommand{\baselinestretch}{1.1}

\newcommand{\tfile}{{\tt hw\_cov2D.ipynb}}

%--------------------------------------------------------------
\begin{document} 
%-------------------------------------------------------------

\begin{spacing}{1.2}
\centering
{\large \bf Problem Set 5: Iterative methods with generalized least squares, Part A [iterA]} \\
\cltag\ \\
Assigned: February 22, 2022 --- Due: March 8, 2022 \\
Last compiled: \today
\end{spacing}

%------------------------

\subsection*{Overview and instructions}

\begin{enumerate}
\item Reading: Section 6.22 of \citet{Tarantola2005}

\item {\bf Problem 1 [Part A].} In \verb+hw_cov+ you explored Gaussian random fields with prescribed covariance (exponential or Gaussian) for a 1D spatial field. This problem deals with 2D Gaussian random fields. Your goal is to understand how these fields are generated, and to get some intuitive sense for how certain parameters change the appearance of the 2D fields.

The codes we use for computing Gaussian random fields in the frequency domain were provided by Miranda Holmes, who wrote them in Matlab as part of the study \citet{BuhlerHolmes2009}. She also wrote a useful set of notes, ``Generating stationary Gaussian random fields,'' which is available as in our google drive notes directory. The Matlab codes have been rewritten for Python (by Yuan Tian) and reside in \verb+lib_fft.py+.

Why do we care about the statistical characteristics of spatial random fields? Because real systems have complex variations that we would like to quantify in simpler terms. \citet{Gneiting2012} examined several different covariance functions, including the Matern family, which includes the Gaussian and exponential functions, and they apply their technique to line transects of arctic sea ice.

\refFig{fig:frankel} provides a motivating figure from \citet{Tarantola2005}.

\item {\bf Problem 2 [Part B].} Iterative methods are needed for nonlinear problems, which are characterized by a nonlinear forward model $\bg(\bem)$. In many problems it is not computationally feasible to evaluate the misfit function dozens of times, let alone millions (or billions) of times needed to cover the $\nparm$-dimensional model space. The basic strategy is to evaluate the misfit function at one point, then use the gradient (and Hessian) at that point to guide the choice of the next point. This problem will build upon the lab exercie \verb+lab_iter.pdf+.

\item {\bf Problem 3 [Part B].} Here we revisit \citet{Tarantola2005}, Problem 7-1. See \verb+hw_epi+ for background.

\end{enumerate}

%------------------------
\clearpage\pagebreak

\begin{figure}
\centering
\includegraphics[width=16cm]{Tarantola2005_Fig5p8.eps}
%
\caption[]
{{
Figure~1 from \citet{FrankelClayton1986}, redisplayed as Figure~5.8 from \citet{Tarantola2005}. Caption: ``Pseudorandom realizations of three types of two-dimensional Gaussian random fields.'' The covariance functions from top to bottom are: Von Karman, exponential, and Gaussian.
\label{fig:frankel}
}}
\end{figure}

%------------------------

\clearpage\pagebreak
\subsection*{Problem 1 (3.0). 2D Gaussian random fields}

In this problem, a ``sample'' is a 2D spatial field that is $n_x \times n_y$.
%
\begin{enumerate}
\item (0.3) Open \tfile, set \verb+run_xy2distance = True+, then run it. Open \verb+lib_fft.py+ to see the function \verb+xy2distance()+.
%
\begin{enumerate}
\item (0.0) Explain how the points are ordered within the distance index matrix \verb+iD+.
\item (0.1) What kind of matrix structure does \verb+iD+ have? (Be as specific as possible.)
\item (0.1) How can you compute the actual distances between points, $\bD$ (or \verb+D+)? 
(Recognize the difference between the distance matrix $\bD$ and the index distance matrix.)
\item (0.1) What is the maximum distance between two points in the example grid?
Note that you need to consider $\Delta x$, which is not represented in the plotted example grid.
\end{enumerate}
%
After you are done, be sure to set \verb+run_xy2distance = False+.

\item (0.0) Use the template code \tfile. Run it. It will will generate 1000 samples of a 2D Gaussian random field with prescribed covariance; eight samples are plotted.

What is the maximum distance between two points in the default grid in \tfile?

\item (0.2) Compute and plot the sample mean $\bmu_{1000}$ with a colorbar.

Throughout this problem, when plotting samples, make sure you do not distort the shape of the 2D samples (\ie aspect ratio 1:1), and use a color scale that ranges between $-3\sigma$ and $+3\sigma$.

%\pagebreak
\item (0.5) Compute the sample covariance matrix $\bC_{1000}$, which we label as \verb+Csamp+.

\begin{enumerate}
\item (0.2) Include a square, colored matrix plot of $\bC_{1000}$.

\item (0.1) Plot the points of $\bC_{1000}$ vs $\bD$. We denote this as $C_{1000}(d)$.

%Hint: Only a single plotting command is needed.

\item (0.1) Superimpose the covariance function $C(d)$.

Hint: \verb+plt.plot(iD*dx,C,'ro')+ is one way to plot $C(d)$, assuming that \verb+iD+ and \verb+C+ are matrices.

\item (0.1) Show that the length scale is consistent with the input value $L'$ (code variable~\verb+Lprime+) (hint: see \verb+hw_cov+ solutions).
\end{enumerate}

\item (0.0) Now set \verb+ifourier=1+ and \verb+idouble=1+ and convince yourself that the FFT method gives the same result. Check that your scatterplot of estimated $C_{1000}(d)$ is about the same.

\item (0.3) The Cholesky decomposition is numerically unstable. In order to consider denser grids, or covariance functions with large $L'$ length scales, we need to use Fourier methods (\verb+ifourier=1+).

\begin{enumerate}
\item (0.1) Explain what happens in the code if \verb+idouble=0+ instead of \verb+idouble=1+.
\item (0.1) Include a plot of $\bC_{1000}$ for \verb+idouble=0+.
\item (0.1) Explain the impact of changing \verb+idouble+ on the samples.
\end{enumerate}
%
After you are done, reset \verb+idouble=1+; from here on out, we will use \verb+idouble=1+.

\item (0.2) Compute the mean and standard deviation of each sample from the set of 1000 samples. Plot the means and standard deviations as two histograms.

\item (0.5) \ptag\ Generate two Gaussian random fields having different covariance (Gaussian or exponential) but using {\em the same Gaussian random vector} (note: this requires coding). Use \verb+nx = 2^5+, \verb+ichol=0+, \verb+ifourier=1+, \verb+idouble=1+, and keep all other parameters fixed (as defaults).

\begin{itemize}
\item Show your modified lines of code.

Hint: Use the template code provided at the end of \tfile\ and see \verb+grf2()+ (in \verb+lib_fft.py+).
%When you generate the first Gaussian random field, save the Gaussian random vectors as \verb+A+ and \verb+B+ in \verb+grf2.m+. Then pass these vectors into \verb+grf2.m+ for the other two sets of samples.

Note: If you do not implement the code correctly, then explain what code modifications are needed.
\end{itemize}

\item \textcolor{red}{\bf Now close your server and start a new one with the large-memory option.}

(0.5)\ptag\ Following the previous problem, set \verb+nx = 2^7+ and then generate three Gaussian random fields: (a) Gaussian covariance, (b) exponential covariance, and (c) circular covariance. In each case, {\em use the same Gaussian random vector}.
%Use \verb+nx = 2^7+, \verb+ichol=0+, \verb+ifourier=1+, \verb+idouble=1+, and keep all other parameters fixed (as defaults).
%
\begin{itemize}
\item Show the GRFs in a $3 \times 1$ subplot figure. \\
In all plots use the same the color range $[-3\sigma, 3\sigma]$ (\verb+plt.clim(3*sigma*np.array([-1,1]))+).  \\
Make sure that you are using \verb+nx = 2^7+.
\end{itemize}

\item (0.5) \ptag\ Choose a covariance function (pick any \verb+icov+), then plot six Gaussian random fields, each with a different length scale $L'$. Use the same Gaussian random vector to generate each GRF. Include a $3 \times 2$ subplot figure showing your GRFs.

\end{enumerate}

%------------------------

\bibliography{carl_abbrev,carl_main,carl_source,carl_him,carl_alaska}
%-------------------------------------------------------------

%\clearpage\pagebreak

%-------------------------------------------------------------

% \begin{figure}
% \centering
% \includegraphics[width=15cm]{covC_LFACTOR2.eps}
% \caption[]
% {{
% Covariance functions from {\tt covC.m} characterized by length scale $L'$ and amplitude $\sigma^2$. The Mat\'ern covariance functions include an additional parameter, $\nu$, that influences the shape: $\nu \rightarrow \infty$ for the Gaussian function (upper left), $\nu = 0.5$ for the exponential function (upper right).
% Some reference e-folding depths are labeled; for example, the $y$-values of the top line is $y = \sigma^2 e^{-1/2} \approx 9.70$.
% \label{fig:covC2}
% }}
% \end{figure}

%-------------------------------------------------------------
\end{document}
%-------------------------------------------------------------
