% dvips -t letter lab_sampling.dvi -o lab_sampling.ps ; ps2pdf lab_sampling.ps
\documentclass[11pt,titlepage,fleqn]{article}

\usepackage{amsmath}
\usepackage{amssymb}
\usepackage{latexsym}
\usepackage[round]{natbib}
%\usepackage{epsfig}
\usepackage{graphicx}
\usepackage{bm}

\usepackage{url}
\usepackage{color}

%--------------------------------------------------------------
%       SPACING COMMANDS (Latex Companion, p. 52)
%--------------------------------------------------------------

\usepackage{setspace}    % double-space or single-space
\usepackage{xspace}

\renewcommand{\baselinestretch}{1.2}

\textwidth 460pt
\textheight 690pt
\oddsidemargin 0pt
\evensidemargin 0pt

% see Latex Companion, p. 85
\voffset     -50pt
\topmargin     0pt
\headsep      20pt
\headheight   15pt
\headheight    0pt
\footskip     30pt
\hoffset       0pt

\include{carlcommands}

\graphicspath{
  {./figures/}
}

\newcommand{\repodir}{{\tt inverse}}

\newcommand{\howmuchtime}{Approximately how much time {\em outside of class and lab time} did you spend on this problem set? Feel free to suggest improvements here.}

% provide space for students to write their solutions
\newcommand{\vertgap}{\vspace{1cm}}

\newcommand{\Ucolor}{\textcolor{red}{\bU}}
\newcommand{\Vcolor}{\textcolor{blue}{\bV}}

\newcommand{\Gcolor}{\textcolor{red}{\bU}\bS\textcolor{blue}{\bV^T}}
\newcommand{\Gpcolor}{\textcolor{red}{\bU_p}\,\bS_p\textcolor{blue}{\bV_p^T}}
\newcommand{\Gdcolor}{\textcolor{blue}{\bV_p}\,\bS_p^{-1}\textcolor{red}{\bU_p^T}}
\newcommand{\GcolorT}{\textcolor{blue}{\bV}\bS^T\textcolor{red}{\bU^T}}
\newcommand{\GpcolorT}{\textcolor{blue}{\bV_p}\,\bS_p\textcolor{red}{\bU_p^T}}
\newcommand{\GdcolorT}{\textcolor{red}{\bU_p}\,\bS_p^{-1}\textcolor{blue}{\bV_p^T}}

\newcommand{\blank}{xxxx}


\renewcommand{\baselinestretch}{1.1}

% change the figures to ``Figure L3'', etc
\renewcommand{\thefigure}{L\arabic{figure}}
\renewcommand{\thetable}{L\arabic{table}}
\renewcommand{\theequation}{L\arabic{equation}}
\renewcommand{\thesection}{L\arabic{section}}

%--------------------------------------------------------------
\begin{document}
%-------------------------------------------------------------

\begin{spacing}{1.2}
\centering
{\large \bf Lab Exercise: The rejection method of generating samples [sampling]} \\
\cltag\ \\
Last compiled: \today
\end{spacing}

%------------------------#

\subsection*{Overview}

\begin{itemize}
\item The purpose of this exercise is to understand one particular method of generating samples from an arbitrary probability distribution. The sampling method, known as the {\em rejection method}, was introduced by \citet{vonNeumann1951} and is also discussed in Section 2.3.2 of \citet{Tarantola2005}. One example of its application can be found in \citet{SilwalTape2016} for the case of a three-dimensional model parameter space defining the orientation of an earthquake source mechanism.

We use the rejection method to generate samples of an arbitrary distribution. As far as I know, there is no such function in Python to do this. (Let me know if you find one!)
%    http://www.mathworks.com/help/stats/common-generation-methods.html#br5k9hi-4

\item A secondary purpose is to become familiar with using custom functions in Python. These are functions that are defined by you within a Python script, such as

\verb+ def p(x): ( A1*np.exp(-(x-x1)**2/(2*sig1**2)) );+

in our lab. These functions can be useful for writing down code that is as close to the mathematics as possible, such as
%
\begin{equation}
p(x) = A_1\,e^{-(x-x_1)^2/(2\sigma_1^2)}
\end{equation}

\item We will also get some practice plotting 2D functions in Python.

\item We will be using the template script \verb+lab_sampling.ipynb+.

%\item Make a copy of the template script:
%
%\begin{verbatim}
%cp lab_sampling_template.m lab_sampling.m
%\end{verbatim}

\end{itemize}

%-------------------------------------------------------------
\bibliography{carl_abbrev,carl_him,carl_main,carl_calif,carl_alaska,carl_source}
%-------------------------------------------------------------

\pagebreak
\subsection*{Exercises}

\begin{enumerate}
\item Review \verb+lab_linefit.ipynb+. What does the Numpy function \verb+np.random.random()+ do? Describe how the histogram plotted achieves its shape.

\vertgap

\item Write a command to return a set of $n$ numbers that are uniformly distributed on the interval $[a,b]$.

\vertgap

\item What does the custom function \verb+plot_histo+ do?

\vertgap

\item Run the example script to generate the figure in \refFig{fig}. The script will generate samples of a specified function that we can think of as a probability density function.

\item Read the code and determine what is being plotted in each subplot of \refFig{fig}. \\
Define $A = \max(p(x))$.
%
\begin{spacing}{2.0}
\begin{enumerate}
\item 
\item 
\item 
\item 
\item 
\item 
\end{enumerate}
\end{spacing}

\item Explain the pattern in \refFig{fig}c and how it occurs. Run the next block of code to annotate (a) to help your understanding.

What fraction of $p(x_i)/A$ are $<$0.05?

What fraction of $p(x_i)/A$ are $>$0.95?

\vertgap

\item Run the next cell. Make sure you understand the relationships among the four subplots. (Note that only the fourth subplot is new.)

\vertgap

\item Explain how a sample ``makes it'' into the final set of accepted samples (\refFig{fig}e).

Why is a low value, $p(x_i)/A < 0.05$, generally going to be rejected?

Why is a high value, $p(x_i)/A > 0.95$, generally going to be kept?

\vertgap

\item What fraction of samples were kept for your run?
% 18825/100000  = 0.188 samples kept

\vertgap

\item Integrate $p(x)/A$ numerically to get the area under the curve. (Hint: You will probably need \verb+dx+) Divide this number by \verb+xmax - xmin+, which represents the area under the curve for the (normalized) uniform distribution. How does this result compare with the fraction of samples that were kept?

\vertgap

\item Now try a more interesting function by changing \verb+ifun=2+ in the 2nd cell block to see the sum of two Gaussians.

What fraction of samples were kept for your run?

What is $\int p(x)/A \, dx$?

\end{enumerate}

%------------------------

\subsection*{Practice}

\begin{enumerate}
\item Now that you've used a function for one variable, try writing one for two variables, $f(x,y)$. This will be useful for Problem 1 of the epicenter homework.

In the space at the bottom of \verb+lab_sampling.ipynb+, write a function for 
%
\begin{equation*}
p(x,y) = A\,e^{-(x^2+y^2)},
\end{equation*}
%
then evaluate the function for some $A$ at some input values $(x_i,y_i)$, and check the output $p(x_i,y_i)$ with the direct calculation of $A\,e^{-(x^2+y^2)}$.

\item If your function is properly defined, then you should be able to plot it using the lines in \verb+lab_sampling.ipynb+. Make sure you understand the three different ways of plotting this function. Note that you can pass any dimensional array (scalar, vector, matrix, etc) as \verb+x+ and \verb+y+ into \verb+p(x,y)+.

\end{enumerate}

% \clearpage\pagebreak

\begin{figure}
\centering
\includegraphics[width=16cm]{lab_sampling_hists1_ifun1.eps}
\caption[]
{{
Figure generated by {\tt lab$\_$sampling.ipynb}. %(\refApp{sec:code}).
Here {\tt xmin = -15} and {\tt xmax = 12}.
\label{fig}
}}
\end{figure}

%------------------------------------------

\clearpage\pagebreak
%\appendix

%\section{Excerpt from {\tt lab$\_$sampling.m}}
%\label{sec:code}

%\tiny
%\begin{spacing}{1.0}
%\begin{verbatim}
% limits for x to consider
% OUTSIDE THESE LIMITS WE ASSUME THAT p(x) = 0
%xmin = -15; xmax = 12;

% define in-line function p(x)
% note that parameters like A1, A2, x1, x2 must be assigned before p()
%x1 = -2; A1 = 2; sig1 = 2;
%x2 =  4; A2 = 1; sig2 = 0.5;
%ifun = 1;
%switch ifun
%    case 1
%        p = @(x) ( A1*exp(-(x-x1).^2/(2*sig1^2)) );
%    case 2
%        p = @(x) ( A1*exp(-(x-x1).^2/(2*sig1^2)) + A2*exp(-(x-x2).^2/(2*sig2^2)) );
%end

% KEY TECHNICAL POINT: f is a function, not a numerical array
% (note that x is not a stored array)
%whos

% analytical curve
%xcurve = linspace(xmin,xmax,1000);
%pcurve = p(xcurve);

% generate samples
% KEY: what does rand do?
%NTRY = 1e5;
%xtry = xmin + (xmax-xmin)*rand(NTRY,1);

% sample the function
%A = max([A1 A2]);           % note: only true for our choice of p(x)
%A = max(pcurve);            % (will work for any densely discretized p(x))
%ptry = p(xtry) / A;         % SET A: values between 0 and 1
%chance = rand(NTRY,1);      % SET B: values between 0 and 1

% plot
%figure; nr=3; nc=2;
%edges1 = [xmin:0.2:xmax]; ne1 = length(edges1);
%edges2 = [0:0.05:1];      ne2 = length(edges2);
%subplot(nr,nc,1); plot(xcurve,pcurve/A);
%xlabel('x'); ylabel('p(x) / A'); title('(a)'); axis([xmin xmax 0 1.2]);
%subplot(nr,nc,2); plot_histo(xtry,edges1); xlim([xmin xmax]);
%xlabel('xtry'); title('(b)'); 
%subplot(nr,nc,3); plot_histo(ptry,edges2);
%xlabel('p(xtry) / A'); title('(c)'); 
%subplot(nr,nc,4); plot_histo(chance,edges2);
%xlabel('chance'); title('(d)'); 

% KEY COMMAND: compare pairs of test samples in sets A and B,
%              then accept or reject the test sample
%ikeep = find(ptry > chance);
%xkeep = xtry(ikeep);

%subplot(nr,nc,5); plot_histo(xkeep,edges1); xlim([xmin xmax]);
%xlabel('xkeep'); title('(e)');

% if p is a probability density and F is the misfit function, then
%    p(x) = exp(-F(x))
%    F(x) = -ln(p(x))
%subplot(nr,nc,6); plot(xcurve,-log(pcurve));
%axis([xmin xmax -1 1.1*max(-log(pcurve))]);
%xlabel('x'); ylabel('F(x) = -ln(p(x))'); title('(f)');
%\end{verbatim}
%\end{spacing}

%-------------------------------------------------------------
\end{document}
%-------------------------------------------------------------
