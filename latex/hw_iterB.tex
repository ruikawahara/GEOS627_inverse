% dvips -t letter hw_iter.dvi -o hw_iter.ps ; ps2pdf hw_iter.ps
\documentclass[11pt,titlepage,fleqn]{article}

\usepackage{amsmath}
\usepackage{amssymb}
\usepackage{latexsym}
\usepackage[round]{natbib}
%\usepackage{epsfig}
\usepackage{graphicx}
\usepackage{bm}

\usepackage{url}
\usepackage{color}

%--------------------------------------------------------------
%       SPACING COMMANDS (Latex Companion, p. 52)
%--------------------------------------------------------------

\usepackage{setspace}    % double-space or single-space
\usepackage{xspace}

\renewcommand{\baselinestretch}{1.2}

\textwidth 460pt
\textheight 690pt
\oddsidemargin 0pt
\evensidemargin 0pt

% see Latex Companion, p. 85
\voffset     -50pt
\topmargin     0pt
\headsep      20pt
\headheight   15pt
\headheight    0pt
\footskip     30pt
\hoffset       0pt

\include{carlcommands}

\graphicspath{
  {./figures/}
}

\newcommand{\repodir}{{\tt inverse}}

\newcommand{\howmuchtime}{Approximately how much time {\em outside of class and lab time} did you spend on this problem set? Feel free to suggest improvements here.}

% provide space for students to write their solutions
\newcommand{\vertgap}{\vspace{1cm}}

\newcommand{\Ucolor}{\textcolor{red}{\bU}}
\newcommand{\Vcolor}{\textcolor{blue}{\bV}}

\newcommand{\Gcolor}{\textcolor{red}{\bU}\bS\textcolor{blue}{\bV^T}}
\newcommand{\Gpcolor}{\textcolor{red}{\bU_p}\,\bS_p\textcolor{blue}{\bV_p^T}}
\newcommand{\Gdcolor}{\textcolor{blue}{\bV_p}\,\bS_p^{-1}\textcolor{red}{\bU_p^T}}
\newcommand{\GcolorT}{\textcolor{blue}{\bV}\bS^T\textcolor{red}{\bU^T}}
\newcommand{\GpcolorT}{\textcolor{blue}{\bV_p}\,\bS_p\textcolor{red}{\bU_p^T}}
\newcommand{\GdcolorT}{\textcolor{red}{\bU_p}\,\bS_p^{-1}\textcolor{blue}{\bV_p^T}}

\newcommand{\blank}{xxxx}


\renewcommand{\baselinestretch}{1.1}

\newcommand{\tfile}{{\tt lab\_iter.ipynb}}
\newcommand{\tfileOM}{{\tt optimization\_method.ipynb}}
\newcommand{\tfileFE}{{\tt forward\_epicenter.ipynb}}
\newcommand{\tfileFEC}{{\tt forward\_epicenter\_crescent.ipynb}}

\graphicspath{
  {./figures/}
  {./seis/figures/}    % only needed when running from inv
}

%--------------------------------------------------------------
\begin{document} 
%-------------------------------------------------------------

\begin{spacing}{1.2}
\centering
{\large \bf Problem Set 5: Iterative methods with generalized least squares, Part B [iterB]} \\
\cltag\ \\
Assigned: February 22, 2022 --- Due: March 8, 2022 \\
Last compiled: \today
\end{spacing}

%------------------------

\subsection*{Background}

See the lab exercise on the quasi-Newton method (\verb+lab_iter.pdf+). There you were asked to write a functioning version of the quasi-Newton algorithm for the 4-parameter epicenter problem. Moving forward, it is important that your code is correct. Start with \verb+lab_iter_sol.ipynb+.
%
Alternatively you can use your \tfile\ script, but, whichever you use, {\bf check your results for the quasi-Newton method with those listed in \refTab{tab}.}

\vspace{0.5cm}
\begin{figure}[h]
\centering
\includegraphics[width=13cm]{optimhw_fig02.eps}
\caption[Source-receiver geometry]
{{
Source--receiver geometry for the earthquake location problem. The ray paths are drawn between the (fixed) receivers and the initial epicenter $\bem_{\rm initial}$.
The target data are computed using the target model \textcolor{red}{$\bem_{\rm target}$}.
The data $\dvec$ are constructed by adding errors to the target data.
Both $\bem_{\rm initial}$ and $\bem_{\rm target}$ are samples within the prior distribution, which is centered at \textcolor{blue}{$\bem_{\rm prior}$} and represented by the 1000 dots. Alternatively, one could choose \textcolor{blue}{$\bem_{\rm prior}$} to be $\bem_{\rm initial}$.
\label{fig:srcrec}
}}
\end{figure}

\clearpage\pagebreak
\subsection*{Problem 2 (4.0). Implementation of iterative methods}

In this problem you will replace the quasi-Newton method (from \verb+lab_iter.pdf+) with three other methods. You will not need to touch the code associated with the forward model (\tfileFE).

\begin{enumerate}
\item (1.5) Implement the {\bf steepest descent} method (Eq. 6.297). Use Eq. 6.309 for $\mu_n$. Use 8~iterations (\verb+niter=8+). Include the following:
%
\begin{enumerate}
\item (1.0) your code
\item (0.1) a plot of the misfit reduction with iteration (note: this plot is produced by default)
\item (0.1) a plot showing epicenter samples of the prior and posterior models (with $\bem_{\rm target}$ and $\bem_{\rm initial}$) (note: this plot is produced by default)
\item (0.3) the posterior model (in \refTab{tab}; list numbers to 0.0001 precision)
\end{enumerate}

\label{steep}

\item (1.5) Repeat Problem 2-1 for the {\bf conjugate gradient} method (Eq. 6.329).
%
\begin{itemize}
\item Use Eq. 6.333 for $\mu_n$.
\item Use Eq. 6.331 for $\alpha_n$.
\item Use $\bF_0 = \bI$ such that $\blambda_n = \bF_0\bgamma_n = \bgamma_n$. 
\item Note that the search direction is initialized as $\bphi_0 = \blambda_0$ ($= \bgamma_0$). (Note that since $\bphi_n = \blambda_n + \alpha_n\bphi_{n-1}$, this implies that $\alpha_0 = 0$.)
\end{itemize}

\item (0.5) \ptag\ Repeat Problem 2-1 for the {\bf variable metric} method (Section 6.22.8).
%
\begin{itemize}
\item Use Eq. 6.333 for $\mu_n$.
\item Use $\bF_0 = \bI$.
\item Use Eq.~6.356 for $\bF_{n+1}$, but note that there is a typo: there should be no transpose on the last $\bdelta\bgamma$ term in the denominator.
\end{itemize}
%
Hint: Write the equations in non-hat notation, such as in Eq. 6.355. For example, note that $\bFh\bgammah = \bF\bgamma$.

\item (0.5)
%
\begin{enumerate}
\item (0.4) Compare and contrast these three methods \citep[see][]{Tarantola2005}.

NOTE: This problem can be answered even if your implementations in 2-1, 2-2, 2-3 were unsuccessful.

\item (0.1) Compare the performance of each method for our problem.
\end{enumerate}

\end{enumerate}

%------------------------

\clearpage\pagebreak
\subsection*{Problem 3 (3.0). Revisiting \citet{Tarantola2005}, Problem 7-1}

Your goal is to apply the quasi-Newton method to Problem 7-1 of \citet{Tarantola2005}.
%
\begin{itemize}
\item Start with a clean version of \tfile\ that replicates the quasi-Newton results shown in \refTab{tab} for \tfileFE.

\item Prepare for this example by copying a file: \\
\verb+cp+ \tfileFE\ \tfileFEC.

Set \verb+iforward=2+ in \tfile\ and then check that your quasi-Newton results with those listed in \refTab{tab}.

%\item \verb+genlsq_crescent.m+ contains the iterative inverse problem, and it also calls the forward model \verb+forward_epicenter.m+. In \verb+genlsq_crescent.m+, replace the call to \verb+forward_epicenter.m+ with \verb+forward_epicenter_crescent.m+.

\end{itemize}

%-----------------------

\begin{enumerate}

\item (2.5) Adapt \tfileFEC\ for Problem 7-1 of \citet{Tarantola2005} (\verb+hw_epi+). \textcolor{red}{\bf In addition to using the values used in Problem 7-1}, make the following additional choices:
%
\begin{itemize}
\item Define $\mprior$ to be the center of the plotting grid used for \verb+hw_epi+. The numbers you want to use are these:
%
\begin{verbatim}
# range of model space (Tarantola Figure 7.1)
xmin = 0
xmax = 22
ymin = -2
ymax = 30
xcen = (xmax+xmin)/2
ycen = (ymin+ymax)/2
\end{verbatim}
%
Then use $\mprior$ = \verb+(xcen, ycen)+.

Use the same $\sigma$ values for the prior epicenter as in \tfileFE\ ($\sigma_{x_s} = \sigma_{y_s} = 10$~km). This prior model is chosen as an analog for the perspective that the epicenter could be anywhere within a large region (such as the plotting grid).

\item $\bem_{\rm initial} = (15, 20)$.

\item $\bem_{\rm target} = (15, 5)$. (This is probably what Tarantola used, though we can't be certain.)

\item For the case of fixed data errors, \verb+tobs+ to be the values used in \verb+hw_epi+:
%
\begin{verbatim}
tobs = np.array([[3.12,3.26,2.98,3.12,2.84,2.98]]).T
eobs = tobs - dtarget
\end{verbatim}
%
This allows us to remove the errors added to the arrival time data listed in Tarantola. (It gets added back in as {\tt dobs = dtarget + eobs}, which is \verb+tobs+.)

\item \verb+axepi = [xmin,xmax,ymin,ymax]+

\end{itemize}

Solve the problem using the quasi-Newton method with eight iterations.
%
\begin{enumerate}
\item (1.8) Include figures showing (a) the misfit reduction; (b) the prior and posterior samples, along with the initial model. (Note that these figures are automatically generated.)

\item (0.5) List the solution after eight iterations: $\bem_{\rm post}$, $\bC_{\rm post}$, and the correlation matrix $\rho_{\rm post}$. Complete \refTab{tab:epi}. 

\item (0.2) How many iterations are needed for convergence?
\end{enumerate}

%----------

\pagebreak
\item (0.5) \ptag\ See \citet[][p.~34--36]{Aster} for how to compute confidence regions. The key concept is that the inequality
%
\begin{equation*}
\left(\bem - \bem_{\rm post} \right)^T \bC_{\rm post}^{-1} \left(\bem - \bem_{\rm post} \right) \le \Delta^2
\end{equation*}
%
describes the interior region of an $\nparm$-dimensional ellipsoid (in our case, $\nparm=2$). For example, $\Delta^2$ can be chosen to represent the boundary of the $95\%$ confidence region.
%
\begin{enumerate}
\item (0.1) Use \verb+eig+ to compute the eigen-decomposition of $\bC_{\rm post}$.

Compute the quantity $\sqrt{\lambda_{\rm max}/\lambda_{\rm min}}$.

\item (0.1) Use \verb+delta2 = chi2inv(0.95,2)+ to compute $\Delta^2$.

Hint: See \verb+ex2p1_ex2p2.ipynb+ for an example of using \verb+chi2inv+, which is in \verb+lib_peip.py+.

\item (0.1) Compute the lengths of the semi-major axis and semi-minor axis of the ellipse, where the length of the $k$th axis is $\Delta\sqrt{\lambda_k}$.

\item (0.2) Plot the confidence region using \verb+plot_ellipse()+, which is a function defined in \tfile. (You should see agreement between the locations of your samples of $\bC_{\rm post}$ and the ellipse.) Also include the ellipse axes in your plot (as well as $\bem_{\rm post}$, $\bem_{\rm initial}$, $\bem_{\rm target}$, samples of prior and posterior, etc).
\end{enumerate}

NOTE: Even if you did not successfully implement \tfileFEC, you can still do the confidence region for the epicenter associated with the forward problem of \tfileFE.

\end{enumerate}


%------------------------

%\pagebreak
\subsection*{Problem} \howmuchtime\

\bibliography{carl_abbrev,carl_main,carl_source,carl_him,carl_alaska}
%-------------------------------------------------------------

\vspace{4cm}

\begin{table}[h]
\centering
\caption[]{
Summary of results for the four iterative methods.
Posterior models are listed for the eighth iteration: $\bem_{\rm post} = \bem_8$.
QN = quasi-Newton; SD = steepest descent; CG = conjugate gradient; VM = variable metric.
\textcolor{red}{List numbers to 0.0001 precision.}
\label{tab}
}
\begin{spacing}{1.4}
\begin{tabular}{r||r|r|r||r|r|r|r}
\hline
& prior & initial & target & QN & SD \hspace{1cm} & CG \hspace{1cm}  & VM \hspace{1cm}  \\
\hline\hline 
$x_s$, km & 35.0000 & 46.5236 & 21.2922 & 20.7327 & & \\ \hline
$y_s$, km & 45.0000 & 40.1182 & 46.2974 & 45.7992 & & \\ \hline
$t_s$, s  & 16.0000 & 15.3890 & 16.1314 & 15.6755 & & \\ \hline
$v$       &  1.6094 &  1.7748 &  2.0903 &  1.9781 & & \\ \hline
\end{tabular}
\end{spacing}
\end{table}

\begin{table}
\centering
\caption[]{
Epicenter problem after eight iterations with the quasi-Newton method.
\textcolor{red}{List numbers to 0.0001 precision.}
\label{tab:epi}
}
\begin{spacing}{1.4}
\begin{tabular}{r||r|r|r||r}
\hline
& prior & initial & target & QN \\
\hline\hline 
$x_s$, km & \hspace{2cm} & \hspace{2cm} & \hspace{2cm} & \hspace{2cm} \\ \hline
$y_s$, km & & & &  \\ \hline
\end{tabular}
\end{spacing}
\end{table}

% \begin{figure}
% \centering
% \includegraphics[width=15cm]{covC_LFACTOR2.eps}
% \caption[]
% {{
% Covariance functions from {\tt covC.m} characterized by length scale $L'$ and amplitude $\sigma^2$. The Mat\'ern covariance functions include an additional parameter, $\nu$, that influences the shape: $\nu \rightarrow \infty$ for the Gaussian function (upper left), $\nu = 0.5$ for the exponential function (upper right).
% Some reference e-folding depths are labeled; for example, the $y$-values of the top line is $y = \sigma^2 e^{-1/2} \approx 9.70$.
% \label{fig:covC2}
% }}
% \end{figure}

%-------------------------------------------------------------
\end{document}
%-------------------------------------------------------------
