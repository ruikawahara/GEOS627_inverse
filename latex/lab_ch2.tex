% dvips -t letter lab_svdgeom.dvi -o lab_svdgeom.ps ; ps2pdf lab_svdgeom.ps
\documentclass[11pt,titlepage,fleqn]{article}

\usepackage{amsmath}
\usepackage{amssymb}
\usepackage{latexsym}
\usepackage[round]{natbib}
%\usepackage{epsfig}
\usepackage{graphicx}
\usepackage{bm}

\usepackage{url}
\usepackage{color}

%--------------------------------------------------------------
%       SPACING COMMANDS (Latex Companion, p. 52)
%--------------------------------------------------------------

\usepackage{setspace}    % double-space or single-space
\usepackage{xspace}

\renewcommand{\baselinestretch}{1.2}

\textwidth 460pt
\textheight 690pt
\oddsidemargin 0pt
\evensidemargin 0pt

% see Latex Companion, p. 85
\voffset     -50pt
\topmargin     0pt
\headsep      20pt
\headheight   15pt
\headheight    0pt
\footskip     30pt
\hoffset       0pt

\include{carlcommands}

\graphicspath{
  {./figures/}
}

\newcommand{\repodir}{{\tt inverse}}

\newcommand{\howmuchtime}{Approximately how much time {\em outside of class and lab time} did you spend on this problem set? Feel free to suggest improvements here.}

% provide space for students to write their solutions
\newcommand{\vertgap}{\vspace{1cm}}

\newcommand{\Ucolor}{\textcolor{red}{\bU}}
\newcommand{\Vcolor}{\textcolor{blue}{\bV}}

\newcommand{\Gcolor}{\textcolor{red}{\bU}\bS\textcolor{blue}{\bV^T}}
\newcommand{\Gpcolor}{\textcolor{red}{\bU_p}\,\bS_p\textcolor{blue}{\bV_p^T}}
\newcommand{\Gdcolor}{\textcolor{blue}{\bV_p}\,\bS_p^{-1}\textcolor{red}{\bU_p^T}}
\newcommand{\GcolorT}{\textcolor{blue}{\bV}\bS^T\textcolor{red}{\bU^T}}
\newcommand{\GpcolorT}{\textcolor{blue}{\bV_p}\,\bS_p\textcolor{red}{\bU_p^T}}
\newcommand{\GdcolorT}{\textcolor{red}{\bU_p}\,\bS_p^{-1}\textcolor{blue}{\bV_p^T}}

\newcommand{\blank}{xxxx}


\renewcommand{\baselinestretch}{1.1}

% change the figures to ``Figure L3'', etc
\renewcommand{\thefigure}{L\arabic{figure}}
\renewcommand{\thetable}{L\arabic{table}}
\renewcommand{\theequation}{L\arabic{equation}}
\renewcommand{\thesection}{L\arabic{section}}

\newcommand{\tfileA}{{\tt ex2p1\_ex2p2.ipynb}}
\newcommand{\tfileB}{{\tt ex3p1.ipynb}}

%--------------------------------------------------------------
\begin{document}
%-------------------------------------------------------------

\begin{spacing}{1.2}
\centering
{\large \bf Lab Exercise: Aster examples, Chapters 2-3 [ch2]} \\
\cltag\ \\
Last compiled: \today \\
%\textcolor{red}{Answer questions in red for lab credit}
\end{spacing}

\subsection*{Overview}

These examples will help prepare you for \verb+hw_ch2.pdf+, which is based on Chapter~2 of \citet{Aster}. The Python notebooks \tfileA\ and \tfileB\ are adapted from the Matlab versions provided with \citet{Aster}.

\subsection*{Part I: Aster Chapter 2 examples}

\begin{enumerate}
\item Review \citet{Aster}, Examples 2-1 and 2-2 by running \tfileA.
(This notebook was needed in \verb+hw_iterB+ for calculating a confidence ellipse for the epicenter crescent misfit problem of \citet{Tarantola2005} in \verb+hw_epi+.)

\item What do you expect to happen to the confidence curves if you decrease \verb+PCONF+?

\item What do you expect to happen to the samples of the posterior if you decrease \verb+PCONF+?

\item Write some lines of code to plot the $m_3$-vs-$m_1$ posterior samples overlain by confidence curves for 10\%, 20\%, \ldots, 90\%.

Suggestion: Start by copying-and-pasting the relevant blocks of code to start a new figure.

\end{enumerate}

%------------------------

\subsection*{Part II: Aster Chapter 3 example}

\begin{enumerate}
\item Read \citet{Aster}, Example 3-1. Then read \tfileB\ and run the script. 
%
\begin{enumerate}
\item What is the forward model in this problem?
\item Examine \refFig{fig:index}. Write the system of equations for $t_1, \ldots, t_8$.
\item Convince yourself that the $\bG$ provided for this problem is correct.
\item Provide a succinct description of this inverse problem.
\end{enumerate}

\item Explain the significance of the spike test results for this problem.

\end{enumerate}

%-------------------------------------------------------------
\bibliography{carl_abbrev,carl_main}
%-------------------------------------------------------------

\begin{figure}[h]
\centering
\begin{tabular}{lc}
{\bf(a)} & \includegraphics[width=8cm]{hw_ch3p2_index.eps} \\
{\bf(b)} & \includegraphics[width=9cm]{aster_tomo_rays.eps} \\
\end{tabular}
\caption[]
{{
Setup for the tomography problem.
(a) Indexing of the $n=9$ model parameters.
(Note that here the checkerboard pattern is only plotted to illuminate the boundaries of the cells.)
(b) Ray paths for the $m=8$ measurements.
\label{fig:index}
}}
\end{figure} 

%-------------------------------------------------------------
\end{document}
%-------------------------------------------------------------
