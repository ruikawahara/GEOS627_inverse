% dvips -t letter lab_svd.dvi -o lab_svd.ps ; ps2pdf lab_svd.ps
\documentclass[11pt,titlepage,fleqn]{article}

\usepackage{amsmath}
\usepackage{amssymb}
\usepackage{latexsym}
\usepackage[round]{natbib}
%\usepackage{epsfig}
\usepackage{graphicx}
\usepackage{bm}

\usepackage{url}
\usepackage{color}

%--------------------------------------------------------------
%       SPACING COMMANDS (Latex Companion, p. 52)
%--------------------------------------------------------------

\usepackage{setspace}    % double-space or single-space
\usepackage{xspace}

\renewcommand{\baselinestretch}{1.2}

\textwidth 460pt
\textheight 690pt
\oddsidemargin 0pt
\evensidemargin 0pt

% see Latex Companion, p. 85
\voffset     -50pt
\topmargin     0pt
\headsep      20pt
\headheight   15pt
\headheight    0pt
\footskip     30pt
\hoffset       0pt

\include{carlcommands}

\graphicspath{
  {./figures/}
}

\newcommand{\repodir}{{\tt inverse}}

\newcommand{\howmuchtime}{Approximately how much time {\em outside of class and lab time} did you spend on this problem set? Feel free to suggest improvements here.}

% provide space for students to write their solutions
\newcommand{\vertgap}{\vspace{1cm}}

\newcommand{\Ucolor}{\textcolor{red}{\bU}}
\newcommand{\Vcolor}{\textcolor{blue}{\bV}}

\newcommand{\Gcolor}{\textcolor{red}{\bU}\bS\textcolor{blue}{\bV^T}}
\newcommand{\Gpcolor}{\textcolor{red}{\bU_p}\,\bS_p\textcolor{blue}{\bV_p^T}}
\newcommand{\Gdcolor}{\textcolor{blue}{\bV_p}\,\bS_p^{-1}\textcolor{red}{\bU_p^T}}
\newcommand{\GcolorT}{\textcolor{blue}{\bV}\bS^T\textcolor{red}{\bU^T}}
\newcommand{\GpcolorT}{\textcolor{blue}{\bV_p}\,\bS_p\textcolor{red}{\bU_p^T}}
\newcommand{\GdcolorT}{\textcolor{red}{\bU_p}\,\bS_p^{-1}\textcolor{blue}{\bV_p^T}}

\newcommand{\blank}{xxxx}


\renewcommand{\baselinestretch}{1.0}

% change the figures to ``Figure L3'', etc
\renewcommand{\thefigure}{L\arabic{figure}}
\renewcommand{\thetable}{L\arabic{table}}
\renewcommand{\theequation}{L\arabic{equation}}
\renewcommand{\thesection}{L\arabic{section}}

%--------------------------------------------------------------
\begin{document}
%-------------------------------------------------------------

\begin{spacing}{1.2}
\centering
{\large \bf Lab Exercise: Singular value spectra [svdspec]} \\
\cltag\ \\
Last compiled: \today
\end{spacing}

%------------------------

\subsection*{Background}

The singular value decomposition is given by
%
\begin{equation*}
\bG = \bU\bS\bV^T
\end{equation*}
%
and is useful for examining linear inverse problems of the form $\bG\bem = \bd$.

The {\em singular value spectrum} is simply the set of singular values
%
\begin{equation*}
s_j = S_{jj}
\end{equation*}
%
where $j$ ranges from 1 to $p$ and $s_p > 0$.

%------------------------

\subsection*{Instructions}

\begin{enumerate}

\item \citet[][p.~74]{Aster} provides criteria for classifying the ``posed-ness'' of a linear inverse problem based on the singular values, $s_j$:
%
\begin{enumerate}
\renewcommand{\theenumi}{\Alph{enumi}}
\item mildly ill-posed: $s_j = k j^{-\alpha}$, $\alpha \le 1$
\item moderately ill-posed: $s_j = k j^{-\alpha}$, $\alpha > 1$
\item severely ill-posed: $s_j = k e^{-\alpha j}$
\end{enumerate}
%
Here $k$ and $\alpha$ are unknown.

Transform each equation so that it can be written as the equation of a line. This should inform how to plot the singular values.

\item The function \verb+ex_getG.m+ will return the forward linear model $\bG$ for four models in \citet{Aster}. Analyze the matrixplots of $\bG$ for each of the 8 examples listed at the top of \verb+ex_getG.m+. Make sure you understand why they look the way they do.

\item Set \verb+bfigure = false+ in \verb+ex_getG.m+, then create a new script that will call \verb+ex_getG.m+ and plot the singular value spectrum for each of the 8 examples.

\item Based the singular value spectra, mark (a), (b), or (c) next to each example below. Matlab's \verb+polyfit+ may be helpful. 
%
\begin{spacing}{1.5}
\begin{verbatim}
G = ex_getG(3,210,210);   % Example 3.2
G = ex_getG(1);           % Example 1.12
G = ex_getG(2,100,100);   % Exercise 1-3-a
G = ex_getG(2,4,4);       % Exercise 1-3-e
G = ex_getG(2,4,20);      % Example 4.4
G = ex_getG(2,20,4);
G = ex_getG(4,20,20);     % Example 1.6, 3.3
G = ex_getG(4,100,100);   % Example 1.6, 3.3
\end{verbatim}
\end{spacing}
%
\end{enumerate}

%-------------------------------------------------------------
\bibliography{carl_abbrev,carl_main}
%-------------------------------------------------------------
\end{document}
%-------------------------------------------------------------
