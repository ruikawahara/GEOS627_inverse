%dvips -t letter hw_grav.dvi -o hw_grav.ps ; ps2pdf hw_grav.ps
\documentclass[11pt,titlepage,fleqn]{article}

\usepackage{amsmath}
\usepackage{amssymb}
\usepackage{latexsym}
\usepackage[round]{natbib}
%\usepackage{epsfig}
\usepackage{graphicx}
\usepackage{bm}

\usepackage{url}
\usepackage{color}

%--------------------------------------------------------------
%       SPACING COMMANDS (Latex Companion, p. 52)
%--------------------------------------------------------------

\usepackage{setspace}    % double-space or single-space
\usepackage{xspace}

\renewcommand{\baselinestretch}{1.2}

\textwidth 460pt
\textheight 690pt
\oddsidemargin 0pt
\evensidemargin 0pt

% see Latex Companion, p. 85
\voffset     -50pt
\topmargin     0pt
\headsep      20pt
\headheight   15pt
\headheight    0pt
\footskip     30pt
\hoffset       0pt

\include{carlcommands}

\graphicspath{
  {./figures/}
}

\newcommand{\repodir}{{\tt inverse}}

\newcommand{\howmuchtime}{Approximately how much time {\em outside of class and lab time} did you spend on this problem set? Feel free to suggest improvements here.}

% provide space for students to write their solutions
\newcommand{\vertgap}{\vspace{1cm}}

\newcommand{\Ucolor}{\textcolor{red}{\bU}}
\newcommand{\Vcolor}{\textcolor{blue}{\bV}}

\newcommand{\Gcolor}{\textcolor{red}{\bU}\bS\textcolor{blue}{\bV^T}}
\newcommand{\Gpcolor}{\textcolor{red}{\bU_p}\,\bS_p\textcolor{blue}{\bV_p^T}}
\newcommand{\Gdcolor}{\textcolor{blue}{\bV_p}\,\bS_p^{-1}\textcolor{red}{\bU_p^T}}
\newcommand{\GcolorT}{\textcolor{blue}{\bV}\bS^T\textcolor{red}{\bU^T}}
\newcommand{\GpcolorT}{\textcolor{blue}{\bV_p}\,\bS_p\textcolor{red}{\bU_p^T}}
\newcommand{\GdcolorT}{\textcolor{red}{\bU_p}\,\bS_p^{-1}\textcolor{blue}{\bV_p^T}}

\newcommand{\blank}{xxxx}


\renewcommand{\baselinestretch}{1.1}

\newcommand{\rh}{\hat{r}}
%\newcommand{\ndata}{m}
%\newcommand{\nparm}{n}
%\newcommand{\ntrun}{p'}

\newcommand{\numdataG}{15}
\newcommand{\numparmG}{14}
\newcommand{\numpG}{14}
\newcommand{\numdataA}{20}
\newcommand{\numparmA}{20}
\newcommand{\numpA}{20}

\newcommand{\mv}{\xi}  % model space discretization
\newcommand{\dv}{x}  % data space discretization
\newcommand{\dep}{h}

%--------------------------------------------------------------
\begin{document}
%-------------------------------------------------------------

\begin{spacing}{1.2}
\centering
{\large \bf Problem Set 7: Regularization by truncated singular value decomposition} \\
\cltag\ \\
Assigned: October 28, 2019 --- Due: November 4, 2019 \\
Last compiled: \today
\end{spacing}

%------------------------

\subsection*{Instructions}

\begin{itemize}
\item Background reading: Ch.~3, \citet{Aster}

Note that the gravity example is rooted in Chapter 1 (Examples 1.4 and 1.5).

\item Relevant example scripts: \verb+ex_3_2_uaf.m+, \verb+ex_3_3_uaf.m+
\item Relevant function: \verb+tsvd.m+ (credit: Regularization Toolbox \citep{Hansen})
\item Template scripts: \verb+grav.m+, \verb+ch3p5.m+
%\item Optional utility scripts: \verb+collocate.m+, \verb+plotconst_mod.m+
%\item Here $\ndata$ is the number of data, and $\nparm$ is the number of model parameters.
\item The \textcolor{red}{the red text sections of Problems 1-\ref{prob1} though 1-\ref{prob2}} denote the questions to be repeated in Problem~2 for the forward model in \refEq{dy}. Points for Problem~2 are worth half what is listed for Problem 1.
\end{itemize}

%==================================

%\pagebreak
\subsection*{Problem 1 (7.0). Gravity surveying for density variations at fixed depth}

Variations of the density of subsurface rock give rise to variations of the gravity field at the Earth's surface. Therefore, from measurements of the gravity field at the Earth surface, one can in principle infer density variations of subsurface rock.

Variations of the vertical component of the gravity field $g(\dv)$ $(0 \le \dv \le 1)$ along a horizontal line at the surface are related to variations $f(\mv)$ $(0 \le \mv \le 1)$ of the mass density along a horizontal line at depth $\dep$ below the surface by the Fredholm integral equation of the first kind \citep[\eg][eq.~1.31]{Aster}
%
\begin{equation}
%g(s) = \int_0^1 K(s,t)\,f(t)\,dt
g(\dv) = \int_0^1 K(\dv,\mv)\,f(\mv)\,d\mv
\label{gs}
\end{equation}
%
with kernel
%
\begin{equation}
K(\dv,\mv) = \frac{\dep}{\left(\dep^2 + (\dv-\mv)^2\right)^{3/2}}
\label{Kxy}
\end{equation}
%
(Dimensional constants such as the gravity constant have been omitted.) In discrete form, we can write the relation between predictions $\bd = (g_1,\ldots,g_\ndata)$ of gravity variations at $\ndata$ points along a line at the surface and variations of the density $\bem = (f_1,\ldots,f_\nparm)$ at $\nparm$ points along a subsurface line as a linear model
%
\begin{equation}
%\bd = \bG\bem + \bepsilon,
\bd = \bG\bem,
\end{equation}
%
%where $\bepsilon$ is a vector of measurement errors, and
the $\ndata \times \nparm$ design matrix $\bG$ is a discrete representation of the integral operator in \refEq{Kxy}.
%The vector $\bem$ of density variations plays the role of the regression coefficients. The vector $\bd$ of gravity variations is the response variable.

The file \verb+grav.m+ provides $\dvec$, the measurements of gravity variations at $\ndata = \numdataG$ equally spaced points along the line $0 \le \dv \le 1$. The standard deviation of the measurement errors is assumed to be $\sigma \approx 0.1$.

You will be asked to compute
%The file \verb+integral.dat+ contains
the $\ndata \times \nparm$ matrix $\bG$ that relates gravity variations $\dvec$ at the $\ndata=\numdataG$ points along the surface to density variations $\bem$ at $\nparm = \numparmG$ points at depth $\dep = 0.25$ below the points of the surface measurements. 

%===================

%\pagebreak

\begin{enumerate}
\item (0.5) Run the template script \verb+grav.m+ to load the data. This template provides the discretization to use for the $\dv_i$ points and $\mv_k$ points.

Make a sketch of the problem in $x$-$z$ space, where $z$ is depth. The aspect ratio need not be one-to-one. Plot and label the $\dv_i$ and $\mv_k$ points in their appropriate locations. (Matlab may be easier to use than sketching by hand.)

%----------

\item (0.2) \textcolor{red}{Explain how the forward model is linear.}
\label{prob1}

%----------

\item (1.0)
\textcolor{red}{
\begin{enumerate}
\item (0.3) Discretize the (integral) forward model (\refeq{gs}).
\item (0.5) Write the system of equations $\bG\bem = \bd$ in matrix schematic form.
\item (0.2) Show your code for computing $\bG$.
\end{enumerate}
}

%----------

\item (0.5) 
\textcolor{red}{
Compute the least-squares solution $\bem_{\rm lsq}$.
\begin{enumerate}
\item (0.1) Plot the predictions from $\bem_{\rm lsq}$ along with the data.
\item (0.2) Separately plot the model.
\item (0.2) Does the least-squares estimate $\bem_{\rm lsq}$ appear to represent plausible variations? (Note that all quantities in this exercise are nondimensional and should be of order $O(1)$.)
\end{enumerate}
}

%----------

\item (2.5)

\begin{enumerate}
\item (0.2) Write the expression for the model vector $\bem_{\ntrun}$ from truncated singular value decomposition, where $\ntrun$ is the truncation parameter in the summation.

Write $\bem_{\ntrun}$ as a matrix expression and as a summation.

\item (0.0) Compute the singular value decoposition.

\item (0.4) \textcolor{red}{Compute the condition number for this system of equations (without using {\tt cond}; show your code and the output value). What does the condition number imply about the influence of noise on your solution?}

\item (0.3) \textcolor{red}{Plot the singular value spectrum on a $\log_{10}$ scale (note \footnote{It doesn't matter whether you use natural log or log-10 ($\log_{10}x = \ln x / \ln 10$), but for the sake of comparison use log-10.}).}

Hint: \verb+plot(log10(svec),'.-')+

\item (0.3) \textcolor{red}{Given that the standard deviation of the measurement error of $\sigma$, what can you infer about the reliability of the least-squares estimate $\bem_{\rm lsq}$?} Explain why the least-squares estimate  $\bem_{\rm lsq}$, the ``best linear unbiased estimate,'' is not a good estimate for~$\bem$.

Hint: What is the SVD expression for the covariance of the solution, $\cov(\bem_{\rm svd})$?

\item (0.5) \textcolor{red}{Compute and plot the Picard ratios $|\bu_k^T\bd|/s_k$, also on a $\log_{10}$ scale.}

\item (0.5) \textcolor{red}{Plot all $\nparm$ basis vectors of model space, $\{\bv_k\}$. Comment on their appearance.} How does the number of zero crossings of the basis vectors behave as a function of the singular value index?

Note: Interpretation is easier if each $\bv_k$ has its own subplot.

\item (0.3) \textcolor{red}{Check that $\{\bv_k\}$ forms an orthonormal basis.}

\end{enumerate}

%----------

\item (1.8)

\begin{enumerate}
\item (0.2) \textcolor{red}{Use TSVD to compute several different solutions, $\bem_{\ntrun}$, for the problem.} \\
Show your Matlab code.

Hint: Use \verb+tsvd.m+.

\item (0.8) \textcolor{red}{Plot an $\ssL$-curve \citep[][p.~95]{Aster} for $\ntrun = 1,\ldots,p$.} \\
Plot both axes on a $\log_{10}$ scale.

\item (0.4) \textcolor{red}{List the $\ntrun$ value that you would pick based on four methods (see table). Briefly justify your choices.}

\begin{center}
\begin{tabular}{l|l}
\hline
method & $\;\;\;\;\ntrun\;\;\;\;$ \\ \hline\hline
singular value spectrum & \\ \hline
Picard ratio spectrum & \\ \hline
$\ssL$-curve & \\ \hline
using $\sigma$ \\ \hline
my preference \\ \hline
\end{tabular}
\end{center}

\item (0.4) \textcolor{red}{Plot your preferred solution $\bem_{\ntrun}$. \\
Separately plot its predictions along with the data.}

\end{enumerate}
\label{prob2}

%----------

\item (0.5) The synthetic ``measurements'' $\bd$ in \verb+gravity.dat+ were generated by discretizing the integral in \refEq{gs}, evaluating it for a mass distribution 
%
\begin{equation}
f(\mv) = \sin(\pi\mv) + 0.5 \sin(2\pi\mv)
\end{equation}
%
and adding pseudo-random noise to the implied gravity variations.

\begin{enumerate}
\item (0.4) Directly compare (in a plot) your regularized estimate $\bem_{\ntrun}$ with the exact solution.
% NOTE: Allow student to either show $f_k = f(\mv_k)$ or $f(\mv)$ as a smooth curve
\item (0.1) Was your $\ntrun$ a good choice?
\end{enumerate}

\end{enumerate}

%------------------------

\pagebreak
\subsection*{Problem 2 (3.0). \ptag\ Aster Problem 3-5 modified}

Run the template script \verb+ch3p5.m+ to get the input data for this problem.
The problem has $\ndata = \numdataA$ data points and $\nparm = \numparmA$ unknown model parameters.
The function $d(\dv)$, $0 \le \dv \le 1$, is related to an unknown function $m(\mv)$, $0 \le \mv \le 1$, by the forward model
%
\begin{equation}
d(\dv) = \int_0^1 \mv\,e^{-\mv \dv}\,m(\mv)\,d\mv.
\label{dy}
\end{equation}
%
The data are listed to four digits only; therefore we can assume that there are ``added errors'' of $\sigma \approx 10^{-4}$.

%-------------

\bigskip\noindent
Complete \textcolor{red}{the red text sections of Problems 1-\ref{prob1} though 1-\ref{prob2}} for the forward model in \refEq{dy}.

%\noindent
%For plotting the discretized models, try to use \verb+plotconst_mod.m+ (see example in \verb+ch3p5.m+); this function is similar to Matlab's \verb+stairs+ function.

%------------------------

%\pagebreak
\subsection*{Problem} \howmuchtime\

\bibliography{carl_abbrev,carl_main,carl_source,carl_him,carl_alaska}

% \clearpage\pagebreak

% \begin{figure}[h]
% \centering
% \includegraphics[width=15cm]{dummy.eps}
% \caption[]
% {{
% Text.
% \label{fig:}
% }}
% \end{figure}

%-------------------------------------------------------------
\end{document}
%-------------------------------------------------------------
