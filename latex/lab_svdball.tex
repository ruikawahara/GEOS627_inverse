% dvips -t letter lab_svdball.dvi -o lab_svdball.ps ; ps2pdf lab_svdball.ps
\documentclass[11pt,titlepage,fleqn]{article}

\usepackage{amsmath}
\usepackage{amssymb}
\usepackage{latexsym}
\usepackage[round]{natbib}
%\usepackage{epsfig}
\usepackage{graphicx}
\usepackage{bm}

\usepackage{url}
\usepackage{color}

%--------------------------------------------------------------
%       SPACING COMMANDS (Latex Companion, p. 52)
%--------------------------------------------------------------

\usepackage{setspace}    % double-space or single-space
\usepackage{xspace}

\renewcommand{\baselinestretch}{1.2}

\textwidth 460pt
\textheight 690pt
\oddsidemargin 0pt
\evensidemargin 0pt

% see Latex Companion, p. 85
\voffset     -50pt
\topmargin     0pt
\headsep      20pt
\headheight   15pt
\headheight    0pt
\footskip     30pt
\hoffset       0pt

\include{carlcommands}

\graphicspath{
  {./figures/}
}

\newcommand{\repodir}{{\tt inverse}}

\newcommand{\howmuchtime}{Approximately how much time {\em outside of class and lab time} did you spend on this problem set? Feel free to suggest improvements here.}

% provide space for students to write their solutions
\newcommand{\vertgap}{\vspace{1cm}}

\newcommand{\Ucolor}{\textcolor{red}{\bU}}
\newcommand{\Vcolor}{\textcolor{blue}{\bV}}

\newcommand{\Gcolor}{\textcolor{red}{\bU}\bS\textcolor{blue}{\bV^T}}
\newcommand{\Gpcolor}{\textcolor{red}{\bU_p}\,\bS_p\textcolor{blue}{\bV_p^T}}
\newcommand{\Gdcolor}{\textcolor{blue}{\bV_p}\,\bS_p^{-1}\textcolor{red}{\bU_p^T}}
\newcommand{\GcolorT}{\textcolor{blue}{\bV}\bS^T\textcolor{red}{\bU^T}}
\newcommand{\GpcolorT}{\textcolor{blue}{\bV_p}\,\bS_p\textcolor{red}{\bU_p^T}}
\newcommand{\GdcolorT}{\textcolor{red}{\bU_p}\,\bS_p^{-1}\textcolor{blue}{\bV_p^T}}

\newcommand{\blank}{xxxx}


\renewcommand{\baselinestretch}{1.1}

% change the figures to ``Figure L3'', etc
\renewcommand{\thefigure}{L\arabic{figure}}
\renewcommand{\thetable}{L\arabic{table}}
\renewcommand{\theequation}{L\arabic{equation}}
\renewcommand{\thesection}{L\arabic{section}}

%--------------------------------------------------------------
\begin{document}
%-------------------------------------------------------------

\begin{spacing}{1.2}
\centering
{\large \bf Lab Exercise: SVD approach to ballistic problem [svdball]} \\
\cltag\ \\
Last compiled: \today
\end{spacing}

%------------------------

\subsection*{Overview}

This is Problem 3 from Chapter 3 \citep{Aster}, revisiting Example~1.1, which was featured in our second problem set (\verb+hw_ch1.pdf+). Review Example~1.1 and the solutions to the problem (if needed). In this lab we apply SVD to the same problem. See Chapter 3 (and notes) for background on SVD.

%------------------------

\subsection*{Aster Chapter 3, Problem 3 (expanded)}

Using the parameter estimation problem described in Example~1.1 for determining the three parameters defining a ballistic trajectory, construct synthetic examples that demonstrate the following four cases using the SVD.

In each case, display and interpret the SVD components $\bU$, $\bV$, and $\bS$ in terms of the rank, $p$, of your forward problem $\bG$ matrix.
Calculate and interpret any model and data null space vector(s).
Calculate model and data space resolution matrices.
%
\begin{enumerate}
\item Write out the forward model for the ballistics problem in matrix schematic form.

\item Exploring the model null space.

\begin{enumerate}
\item Three data points ($\ndata = 3$) that are exactly fit by a unique model (\refFig{fig:unique}). \\
Plot your data points and the predicted data for your model.

\item Two data points ($\ndata = 2$) that are exactly fit by an infinite suite of parabolas. \\
Plot your data points and the predicted data for a suite of these models.

\item Four data points ($\ndata = 4$) that are only approximately fit by a parabola. \\
Plot your data points and the predicted data for the least squares model.

Choose one of the sets of data in \refFig{fig:unique}. Add a fourth point that is off the curve. Then plot the updated solution as a new parabola. Does the new parabola fit any of the points?

\item Two data points ($\ndata = 2$) that are only approximately fit by any parabola, and for which there are an infinite number of least squares solutions. \\
Plot your data points and the predicted data for a suite of least squares models.
\end{enumerate}

\item Exploring the data null space.

\begin{enumerate}
\item For the case of $\ndata = 4$ fit approximately by a parabola, use the data null space vector(s) to generate other data sets whose solutions are the same parabola. What can you say about the trade-offs in $y$ values among the data points?
\end{enumerate}

\end{enumerate}

%-------------------------------------------------------------
\bibliography{carl_abbrev,carl_main}
%-------------------------------------------------------------

\clearpage\pagebreak
\begin{figure}
\centering
\includegraphics[width=16cm]{svdball_uniqueexact}
\caption[]
{{
Examples of sets of $\ndata = 3$ for which there are unique solutions.
\label{fig:unique}
}}
\end{figure}

%-------------------------------------------------------------
\end{document}
%-------------------------------------------------------------
